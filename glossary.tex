\newglossaryentry{dokumentaatio}{
	name=dokumentaatio,
	description={(engl. \textit{documentation}) Ohjelmiston tai ohjelmointikielen hakuteosta muistuttava asiakirja, joka kertoo yksityiskohtaisesti sen ominaisuuksista. Vaatii yleensä esitietoja. Pythonin dokumentaatio löytyy osoitteesta \url{https://docs.python.org/3/}}
}

\newglossaryentry{komentotulkki}{
	name=komentotulkki,
	description={(engl. \textit{shell} tai \code{interpreter}) Interaktiivinen ohjelma, johon käyttäjä syöttää ohjelmakoodia, joka suoritetaan välittömästi. Python-komentotulkin saa auki \code{python}-komennolla; myös IDLE:ssä on komentotulkki}
}

\newglossaryentry{IDLE}{
	name=IDLE,
	description={(Integrated Development and Learning Environment) Helppokäyttöinen ohjelma Python-koodin käsittelemiseen}
}

\newglossaryentry{lause}{
	name=lause,
	description={(engl. \textit{statement}) Sellainen pätkä Python-koodia, joka voi esiintyä itsenäisesti. Esimerkiksi \code{print("kissa")} tai \code{x=6} ovat lauseita}
}

\newglossaryentry{funktio}{
	name=funktio,
	description={(engl. \textit{function}) Ohjelmoinnissa sellainen arvo, jota voidaan kutsua antamalla sille nolla tai useampi argumenttia. Funktioilla voi olla paluuarvo, sivuvaikutuksia tai ei kumpaakaan. Esimerkiksi \code{print} on funktio, joka tulostaa sille annetun argumentin}
}

\newglossaryentry{merkkijono}{
	name=merkkijono,
	description={(engl. \textit{string}) Merkeistä koostuva pätkä tekstiä. Pythonissa merkkijonoja voi merkitä asettamalla ne lainausmerkkien sisään: esimerkiksi \code{"kissa"} on merkkijono}
}

\newglossaryentry{versiohallintaohjelma}{
	name=versiohallintaohjelma,
	description={(engl. \textit{version control system}) Koodin säilyttämiseen ja historiatietojen kirjaamiseen suunniteltu ohjelmisto, joka helpottaa useamman ohjelmoijan yhteistyötä. Suosittuja ovat nykyisin mm. Git ja Mercurial}
}

\newglossaryentry{koodinvaihtomerkki}{
	name=koodinvaihtomerkki,
	description={(engl. \textit{escape character}) Koodinvaihtomerkki on jokin merkki (Pythonissa kenoviiva \code{\textbackslash}), jonka avulla voidaan kirjoittaa merkkejä, jotka muuten tulkittaisiin virheellisesti. Esimerkiksi merkkijonojen sisällä lainausmerkin saa kirjoittamalla \code{\textbackslash"}, sillä pelkkä lainausmerkki tulkittaisiin merkkijonon päättymiseksi}
}

\newglossaryentry{muuttuja}{
	name=muuttuja,
	description={(engl. \textit{variable}) \glslink{lauseke}{Lauseke}, joka viittaa sille aiempin määriteltyyn arvoon. Muuttujien nimillä on joitakin rajoituksia; \code{x3} ja \code{var} ovat sallittuja, mutta \code{muut)} ja \code{4y} ovat kiellettyjä}
}

\newglossaryentry{tyyppi}{
	name=tyyppi,
	description={(engl. \textit{type}) Jokaisella arvolla on tyyppi, joka kertoo sen ominaisuudet, kuten sen, mitä operaattoreita ja metodeja sillä on}
}

\newglossaryentry{liukuluku}{
	name=liukuluku,
	description={(engl. \textit{floating point number}) Pythonin vastine desimaaliluvuille. Liukuluvuilla laskeminen on niiden sisäisestä esityksestä johtuen epätarkkaa}
}

\newglossaryentry{sisennys}{
	name=sisennys,
	description={(engl. \textit{indentation}) Rivin alussa olevien välilyöntien tai tabulaattorien määrä}
}

\newglossaryentry{syntaksi}{
	name=syntaksi,
	description={(engl. \textit{syntax}) Ohjelmointikielen kielioppi eli se, minkälaisia osia koodista on sallittua käyttää missäkin yhteydessä. Vertaa \glslink{semantiikka}{semantiikkaan}}
}

\newglossaryentry{semantiikka}{
	name=semantiikka,
	description={(engl. \textit{semantics}) Termi sille, mikä merkitys koodin eri osilla on. Vertaa \glslink{syntaksi}{syntaksiin}}
}

\newglossaryentry{funktiokutsu}{
	name=funktiokutsu,
	description={(engl. \textit{function call}) Eräs \gls{lauseke}, jossa on tietyn \glslink{funktio}{funktion} nimi sekä suluin ympäröidyt argumentit. Esimerkiksi \code{print("Hello World!")} on funktiokutsu}
}

\newglossaryentry{lauseke}{
	name=lauseke,
	description={(engl. \textit{expression}) Osa koodia, jolla on jollakin \glslink{tyyppi}{tyypillä} kuvattava arvo. Esimerkiksi \code{2+4}, \code{"merkkijono"} ja \code{int("66")} ovat lausekkeita}
}

\newglossaryentry{sivuvaikutus}{
	name=sivuvaikutus,
	description={(engl. \textit{side effect}) Sellainen toiminta, jolla \gls{funktio} vaikuttaa ohjelman tilaan ja tekee jotain muutakin kuin vain laskee paluuarvonsa}
}

\newglossaryentry{rekursio}{
	name=rekursio,
	description={(engl. \textit{recursion}) Se, että \gls{funktio} kutsuu itseään}
}

\newglossaryentry{avainsana}{
	name=avainsana,
	description={(engl. \textit{keyword}) Pythonin varaama termi, kuten \code{if} tai \code{while}, jota ei saa käyttää \glslink{muuttuja}{muuttujan} nimenä}
}

\newglossaryentry{tietorakenne}{
	name=tietorakenne,
	description={(engl. \textit{data structure}) Arvo, joka voi sisältää muita arvoja. Esimerkiksi \gls{lista} on tietorakenne}
}

\newglossaryentry{lista}{
	name=lista,
	description={(engl. \textit{list}) \glslink{indeksoitu}, muuttuva \gls{tietorakenne}, jonka voi määritellä syntaksilla \code{[1, 2, 3]}}
}

\newglossaryentry{alkio}{
	name=alkio,
	description={(engl. \textit{element, item, member}) Jonkin \glslink{tietorakenne}{tietorakenteen} sisältämä arvo. Esimerkiksi listan \code{["a", "b"]} alkioita ovat \code{"a"} ja \code{"b"}}
}

\newglossaryentry{joukko}{
	name=joukko,
	description={(engl. \textit{set}) \glslink{tietorakenne}{Tietorakenne}, jonka \glslink{alkio}{alkiot} eivät ole järjestyksessä, ja jossa jokaista alkiota voi olla vain yksi kappale}
}

\newglossaryentry{tuple}{
	name=tuple,
	description={(engl. \textit{tuple}, suomeksi myös \textit{monikko} ja \textit{tuppeli}) Muuttumaton, \gls{indeksoitu} \gls{tietorakenne}. Vertaa \glslink{lista}{listaan}}
}

\newglossaryentry{indeksoitu}{
	name=indeksoitu,
	description={(engl. \textit{ordered, indexed}) \glslink{tietorakenne}{Tietorakenne} on indeksoitu, jos sen \glslink{alkio}{alkiot} ovat jossakin järjestyksessä. Esimerkiksi \glslink{lista}{listat} \code{[4, 5]} ja \code{[5, 4]} ovat eri listoja, koska niiden alkiot ovat eri järjestyksessä, joten lista on indeksoitu tietorakenne}
}

\newglossaryentry{hajautustaulu}{
	name=hajautustaulu,
	description={(engl. \textit{hash table, hash map, dictionary}) \glslink{tietorakenne}{Tietorakenne}, jossa jokaista avainta vastaa jokin arvo. Avaimet ja arvot voivat olla mitä tahansa tyyppiä}
}
